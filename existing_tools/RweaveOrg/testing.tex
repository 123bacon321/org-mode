% Created 2009-02-08 Sun 11:18
\documentclass[11pt]{article}
\usepackage[utf8]{inputenc}
\usepackage[T1]{fontenc}
\usepackage{graphicx}
\usepackage{longtable}
\usepackage{hyperref}
\usepackage{Sweave}

\title{testing}
\author{Dan}
\date{08 February 2009}

\begin{document}

\maketitle

\setcounter{tocdepth}{3}
\tableofcontents
\vspace*{1cm}

\section{Sweave and org-mode}
\label{sec-1}

  If you're reading a PDF version of this document, you should also
  look at \href{file:///home/dan/src/rorg/existing_tools/RweaveOrg/testing.Rorg}{testing.Rorg} (the source file) and \href{file:///home/dan/src/rorg/existing_tools/RweaveOrg/testing.org}{testing.org} (the output
  of the Sweave process).

  Keep in mind that one of the advantages of a block-based approach is
  using \texttt{C-'} to edit code in its native mode.

\subsection{R code that is not printed}
\label{sec-1.3}


\subsection{R code that is printed}
\label{sec-1.4}

\begin{Schunk}
\begin{Sinput}
> c <- 4
\end{Sinput}
\end{Schunk}
   
   We can use block labels to embed blocks by reference (even if they
   weren't printed before).
\subsection{R code that references other blocks}
\label{sec-1.5}

\begin{Schunk}
\begin{Sinput}
> a <- 3
> b <- 6
> c <- 4
> a + b + c
\end{Sinput}
\begin{Soutput}
[1] 13
\end{Soutput}
\end{Schunk}

\subsection{Used in text}
\label{sec-1.7}

    The value of \verb=a= is 3.

\subsection{Used in a table}
\label{sec-1.8}


\begin{center}
\begin{tabular}{rrrr}
 a  &  b  &  c  &  TOTAL  \\
\hline
 3  &  6  &  4  &     13  \\
\end{tabular}
\end{center}




\end{document}
